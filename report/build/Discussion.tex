%What are the strengths and shortcomings of your device? Did it match the requirements? How would you improve/develop it further, if you had time? If you had to produce your device in a factory for mass production, what would you modify?
The robot is able to be remote controlled via the IOS app and avoids user error with obstacle detection via the ultrasonic sensor. With the addition of actuated feet, the robots performance was further improved enabling it to walk on rougher and more uneven surfaces. One major issue in regards to the robot is the difficulty in assembly, as one of the original goals of the project was to reduce weight, the overall footprint of the robot is relatively small. This results in the usage of m2 screws for assembly of the legs, resulting in finicky and frustrating assembly. Even given the relatively small footprint, friction is still the primary concern, and the point that could be improved the most. In a future iteration of the project, the usage of bearing would further decrease friction, perhaps enabling us to reduce the gear ratio, to increase speed. In addition, print-in-place joints for 3d-print manufacturing could be an interesting solution for leg assembly, as the entire leg would be printed as single object, something only enabled by usage of 3d-printing. \\ \\ For mass production the highest priority would be to ease the assembly of the legs, as such the aforementioned usage of print in place joints could be used to print the legs as a single object. To coalesce the electronics on the base and remove the breadboard, one could also develop a PCB to connect the different devices, in this case we could forego the expensive motor shield and instead use a motor driver, soldered into the pcb. This approach would be cheaper, while still eleminating wires.